% فایل LaTeX برای پروژه: سامانه هوشمند پیش‌بینی قیمت ملک
% برای کامپایل با XeLaTeX توصیه می‌شود
\documentclass[12pt,a4paper]{article}

% بسته‌های مهم
%\documentclass{article}
\usepackage{makecell}


\usepackage{lipsum}
\usepackage{fontspec}      % برای فونت‌ها (XeLaTeX)
\usepackage{graphicx}
\usepackage{geometry}
\usepackage{longtable}
\usepackage{tabularx}
\usepackage{tabularray}     % conflict
\renewcommand\theadfont{\bfseries}
\renewcommand\theadgape{}
\newcolumntype{Y}{>{\centering\arraybackslash}X}
%\usepackage{tblr}           % conflict
\usepackage{booktabs}
\usepackage{hyperref}
\usepackage{enumitem}
\usepackage{multicol}
\usepackage{tikz}
\usetikzlibrary{positioning,shapes,arrows.meta}
\usepackage{float}
\usepackage{xepersian}     % پشتیبانی از فارسی

% تنظیم فونت (اگر فونت نصب نیست، نام فونت را تغییر دهید یا این خطوط را حذف کنید)
\settextfont{Amiri}
% اگر می‌خواهید از فونت لاتین خاصی استفاده کنید، می‌توانید از این دستور استفاده کنید:
% \setlatintextfont{Times New Roman}

\geometry{top=2.5cm, bottom=2.5cm, left=2.5cm, right=2.5cm}
\hypersetup{
	pdftitle={سامانه هوشمند پیش‌بینی قیمت ملک},
	pdfauthor={Amir J. Khafaji},
	unicode=true,
	colorlinks=true,
	linkcolor=blue,
	citecolor=blue,
	urlcolor=blue
}

% عنوان
\title{سامانه هوشمند پیش‌بینی قیمت ملک و مشاوره ساخت‌وساز}
\author{تهیه‌کننده: Amir J. Khafaji}
\date{\today}

\begin{document}
	%\maketitle
	\begin{center}
		\rule{0.9\textwidth}{0.5pt}
	\end{center}
	
	\begin{abstract}
		این سند شرح کامل پروژه «سامانه هوشمند پیش‌بینی قیمت ملک و مشاوره ساخت‌وساز» شامل اهداف، توصیف سیستم، مزایا، چالش‌ها، توجیه اقتصادی، بودجه، زمان‌بندی، فازبندی، نیازمندی‌های پرسنلی، قرارداد نمونه پرسنل و پیوست‌هایی مانند جداول و نمودارها می‌باشد. این متن برای ارائه به سرمایه‌گذار یا تیم توسعه آماده شده است.
	\end{abstract}
	
	\tableofcontents
	\newpage
	
	\section{مقدمه}
	هدف این پروژه طراحی و پیاده‌سازی یک سامانه هوشمند برای پیش‌بینی قیمت املاک و ارائه خدمات تکمیلی مرتبط (پیشنهاد املاک مشابه، پیش‌بینی بلندمدت قیمت، مشاوره ساخت‌وساز و گزارش‌های تحلیلی) می‌باشد. سامانه بر پایه مدل‌های یادگیری ماشین و یادگیری عمیق توسعه یافته و داده‌ها از منابع متعدد (آگهی‌ها، ثبت معاملات، داده‌های مکانی، تصاویر ملک) گردآوری می‌شوند.
	
	\section{تعریف مسئله و اهداف}
	هدف اصلی ارائه تخمین دقیق قیمت و محدوده قیمت برای املاک است تا خریداران، فروشندگان و مشاوران املاک بتوانند تصمیمات آگاهانه‌تری اتخاذ کنند. اهداف فرعی عبارتند از:
	\begin{itemize}
		\item ارائه محدوده قیمت (Min -- Avg -- Max) برای هر ملک
		\item پیشنهاد املاک مشابه بر اساس ویژگی‌های کلیدی
		\item پیش‌بینی روند قیمت در بازه‌های میان‌مدت و بلندمدت
		\item ارائه داشبورد تحلیلی برای مشاوران املاک
		\item درآمدزایی از طریق اشتراک و درصد از معاملات
	\end{itemize}
	
	\section{توصیف سیستم}
	\subsection{ورودی‌ها}
	\begin{itemize}
		\item ویژگی‌های ساختاری: متراژ، تعداد اتاق خواب، سال ساخت، کیفیت ساخت، نقشه
		\item ویژگی‌های مکانی: منطقه، نزدیکی به امکانات، دسترسی به حمل‌ونقل
		\item تصاویر ملک (عکس‌های داخلی/نمای بیرونی)
		\item داده‌های بازار: قیمت‌های ثبت‌شده روزانه، معاملات گذشته
		\item اطلاعات تکمیلی: طبقه، پارکینگ، آسانسور و ...
	\end{itemize}
	
	\subsection{خروجی‌ها}
	\begin{itemize}
		\item قیمت پیشنهادی و بازه قیمت
		\item پیشنهاد املاک مشابه با فیلترهای قابل تنظیم
		\item نمودار روند قیمت و پیش‌بینی سالیانه
		\item گزارش‌های قابل دانلود برای مشتریان و سرمایه‌گذاران
	\end{itemize}
	
	\subsection{معماری کلی (سطح بالا)}
	معماری پیشنهادی شامل اجزای زیر است:
	\begin{enumerate}
		\item \textbf{لایه داده}: پایگاه داده معاملات، تصاویر، متادیتا
		\item \textbf{لایه پردازش داده}: پاک‌سازی، دسته‌بندی، ETL
		\item \textbf{لایه مدلسازی}: مدل‌های یادگیری ماشین و شبکه‌های عصبی برای قیمت و تحلیل تصویر
		\item \textbf{بک‌اند / API}: سرویس‌های RESTful برای درخواست‌ها
		\item \textbf{فرانت‌اند}: داشبورد وب و اپلیکیشن موبایل
		\item \textbf{مدیریت و گزارش‌دهی}: ابزارهای BI و گزارش‌ساز
	\end{enumerate}
	
	\section{مزایا}
	\begin{itemize}
		\item دقت بالاتر در قیمت‌گذاری نسبت به روش‌های سنتی
		\item تسریع ارتباط بین خریدار و فروشنده
		\item کاهش نیاز به بازدیدهای حضوری و افزایش کارایی
		\item هشدار به‌موقع برای قیمت‌گذاری غیرواقعی
		\item قابلیت گسترش بین‌المللی
		\item رابط کاربری ساده و نگهداری آسان
	\end{itemize}
	
	\section{چالش‌ها و ریسک‌ها}
	\begin{itemize}
		\item نیاز به مدل تحلیل تصویر و داده‌های برچسب‌خورده
		\item کمبود داده‌های معتبر و کیفیت پایین داده‌ها
		\item هزینه سخت‌افزاری بالا (GPU/سرورها)
		\item هزینه و زمان توسعه و جذب نیروی انسانی متخصص
		\item مسائل حقوقی و رعایت مقررات محلی در دسترسی به داده‌ها
		\item نیاز به عملیات جمع‌آوری داده حضوری در برخی مناطق
	\end{itemize}
	
	\section{توجیه اقتصادی و مدل درآمد}
	\subsection{روش‌های درآمدزایی}
	\begin{itemize}
		\item دریافت درصد از هر معامله (مثلاً ۰.۵\% تا ۲\%)
		\item فروش اشتراک‌های ماهانه/سالیانه برای مشاوران حرفه‌ای
		\item فروش API و دسترسی داده به شرکت‌ها
		\item ارائه گزارش‌های تحلیلی سفارشی و مشاوره ساخت‌وساز با هزینه
	\end{itemize}
	
	\subsection{بودجه و برآورد کل}
	\begin{itemize}
		\item بودجه کل مورد نیاز: \textbf{۳,۰۰۰,۰۰۰,۰۰۰} تومان
		\item مدت زمان توسعه: ۱۲ ماه
	\end{itemize}
	
	\begin{table}[H]
		\centering
		\caption{Expenses}
		\begin{tabularx}{\textwidth}{l X r}
			\toprule
			Row & Obbject & Cost(Toman) \\
			\midrule
			1 & Salary & 1,200,000,000 \\
			2 & Servers and GPUs & 600,000,000 \\
			3 & Data Gathering & 300,000,000 \\
			4 & UI/UX & 250,000,000 \\
			5 & Ads & 200,000,000 \\
			6 & Licences and State Costs & 100,000,000 \\
			7 & Save  & 350,000,000 \\
			\bottomrule	
		\end{tabularx}
	\end{table}
	
	\section{پرسنل مورد نیاز}
	\begin{itemize}
		\item ۲ مهندس یادگیری ماشین / هوش مصنوعی
		\item ۲ تحلیل‌گر/مهندس داده
		\item ۱ توسعه‌دهنده بک‌اند
		\item ۱ توسعه‌دهنده فرانت‌اند
		\item ۱ طراح UI/UX
		\item تیم جمع‌آوری داده (۳–۵ نفر)
		\item مدیر پروژه
	\end{itemize}
	
	\section{فازبندی پیشنهادی}
	\begin{longtable}{p{2cm} p{8cm} p{3cm}}
		\toprule
		فاز & شرح & مدت (ماه) \\
		\midrule
		فاز ۱ & جمع‌آوری داده، طراحی ساختار دیتابیس، تعریف ویژگی‌ها & ۲ \\
		فاز ۲ & توسعه مدل‌های قیمت و تحلیل تصویر، نمونه‌سازی (prototype) & ۳ \\
		فاز ۳ & توسعه بک‌اند و APIها & ۲ \\
		فاز ۴ & طراحی و توسعه رابط کاربری (وب و موبایل) & ۲ \\
		فاز ۵ & تست، اعتبارسنجی و بهینه‌سازی مدل‌ها & ۲ \\
		فاز ۶ & لانچ، بازاریابی و پشتیبانی اولیه & ۱ \\
		\bottomrule
	\end{longtable}
	
	\section{ایده‌های توسعه و ماژول‌های اختیاری}
	\begin{itemize}
		\item سیستم نمره‌دهی نقشه ملک (Scoring Plan)
		\item تعیین ضریب تأثیر منطقه و دسترسی‌ها بر قیمت
		\item تعیین ضریب کیفیت ساخت بر قیمت
		\item سامانه قرارداد آنلاین بین مشتری و پرسنل
		\item پرسشنامه هوشمند برای جمع‌آوری نیازهای مشتری
		\item محصولات آموزشی و پشتیبانی درون‌گروهی
	\end{itemize}
	
	\section{نمونه پرسشنامه کوتاه برای مشتری}
	\begin{enumerate}
		\item بودجه تقریبی شما چقدر است؟
		\item متراژ مطلوب (متر مربع):
		\item تعداد اتاق خواب موردنظر:
		\item محل/منطقه ترجیحی:
		\item اولویت‌ها (پارکینگ، آسانسور، تراس، نورگیر و ...):
		\item آیا زمان خاصی برای نقل‌مکان مدنظر دارید؟
	\end{enumerate}
	
	\section{قرارداد همکاری پرسنل (پیش‌نویس)}
	\subsection*{ماده اول: موضوع قرارداد}
	پرسنل مطابق پوزیشن کاری و وظایف مشخص‌شده در پروژه/شرکت مشغول به کار می‌شود.
	
	\subsection*{ماده دوم: مدت قرارداد}
	این قرارداد از تاریخ \underline{\hspace{4cm}} تا تاریخ \underline{\hspace{4cm}} معتبر است و قابل تمدید می‌باشد.
	
	\subsection*{ماده سوم: تعهدات پرسنل}
	\begin{itemize}
		\item انجام تعهدات کاری طبق شرح وظایف (ساعتی یا کاری)
		\item عدم افشای اطلاعات کد، داده‌ها و مستندات شرکت
		\item عدم انجام پروژه مشابه خارج از چارچوب قرارداد
	\end{itemize}
	
	\subsection*{ماده چهارم: تعهدات کارفرما}
	\begin{itemize}
		\item پرداخت حقوق و مزایا در ۳ روز ابتدایی هر ماه
		\item تأمین اطلاعات و ابزارهای لازم برای انجام کار
		\item حفظ حقوق معنوی و ایجاد فضای کاری سالم
	\end{itemize}
	
	\subsection*{ماده پنجم: فسخ قرارداد}
	\begin{itemize}
		\item در صورت فسخ از سوی پرسنل: پرداخت معادل دو ماه حقوق به کارفرما و همکاری تا جایگزینی نیروی جدید (حداکثر دو ماه)
		\item در صورت فسخ از سوی کارفرما: پرداخت معادل دو ماه حقوق و مزایا به پرسنل
		\item در صورت تخلف پرسنل از قوانین یا مفاد قرارداد، کارفرما مجاز به فسخ بدون جریمه است
	\end{itemize}
	
	\subsection*{ماده ششم: امضا و تعهد}
	این قرارداد در دو نسخه تنظیم و هر دو نسخه معتبر می‌باشند.
	\\
	\noindent\textbf{امضای کارفرما:} \rule{6cm}{0.4pt} \qquad \textbf{امضای پرسنل:} \rule{6cm}{0.4pt}
	
	\section{فرم‌های تست و اعتبارسنجی}
	موارد قابل تست:
	\begin{itemize}
		\item شاخص دقت مدل (MAE, RMSE, MAPE)
		\item تست تطابق قیمت با معاملات واقعی در دوره‌های گذشته
		\item تست عملکرد مدل تحلیل تصویر (دقت تشخیص ویژگی‌ها)
		\item تست کارایی API (تأخیر پاسخ، تحمل بار)
		\item تست قابلیت استفاده رابط کاربری (Usability)
	\end{itemize}
	
	\section{برآورد اولیه هزینه ساخت‌وساز}
	در صورت نیاز به برآورد هزینه ساخت به ازای هر متر مربع، می‌توان جداول استاندارد مصالح، هزینه نیروی انسانی و هزینه‌های جانبی را به‌تفصیل وارد نمود.
	
	\section{پیوست‌ها}
	\subsection{چارت فلوچارت (نمونه)}
	\begin{figure}[H]
		\centering
		\begin{tikzpicture}[node distance=1.2cm, every node/.style={draw, rectangle, rounded corners}]
			\node (start) {شروع: ورود اطلاعات ملک};
			\node (clean) [below=of start] {پاک‌سازی و پیش‌پردازش داده};
			\node (model) [below=of clean] {مدلسازی (قیمت + تصویر)};
			\node (api) [below=of model] {پیشنهادات و تولید گزارش};
			\node (end) [below=of api] {خروجی: قیمت و گزارش};
			\draw[->] (start) -- (clean) -- (model) -- (api) -- (end);
		\end{tikzpicture}
		\caption{فلوچارت ساده فرآیند}
	\end{figure}
	
	\subsection{جدول فازها (خلاصه)}
	\begin{table}[H]
		\centering
		\caption{Project Timetable}
		\begin{tabular}{l c}
			\toprule
			Phase & Months \\
			\midrule
			Data Gathering & 2 \\
			Developing Model & 2 \\
			Backend and API & 2 \\
			Front and UI & 2 \\
			Test and Optimization & 2 \\
			Start and Marketing & 1 \\
			\bottomrule
		\end{tabular}
	\end{table}
	
	\section{نکات تکمیلی برای تیم توسعه}
	\begin{itemize}
		\item روش پیشنهادی جمع‌آوری داده: ترکیب داده‌های آگهی عمومی، همکاری با شرکت‌های ثبت معاملات و عملیات میدانی
		\item توصیه برای مدلسازی: شروع با مدل‌های کلاسیک (رگرسیون، XGBoost) و سپس گسترش به شبکه‌های عصبی برای تحلیل تصویر و ترکیب ویژگی‌ها (Multimodal)
		\item پیشنهاد تست: تقسیم زمانی داده‌ها (Time-based split) برای جلوگیری از نشت اطلاعات (data leakage)
		\item مستندسازی کد و APIها را از ابتدای پروژه الزامی کنید
	\end{itemize}
	
	\section*{خاتمه}
	این سند یک پیش‌نویس کامل برای ارائه به سرمایه‌گذار یا تیم توسعه می‌باشد. در صورت نیاز می‌توان فایل را به فرمت PDF تبدیل و جداول/نمودارها را با اعداد دقیق‌تر و نمودارهای گرافیکی تکمیل کرد.
	
	\vspace{1cm}
	\noindent\textbf{تهیه‌کننده:} Amir J. Khafaji\\
	\noindent\textbf{تاریخ:} \today
	
\end{document}
